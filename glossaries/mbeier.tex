\newacronym{db}{DB}{Datenbank}
\newacronym{dbms}{DBMS}{Datenbankmanagementsystem}

\newglossaryentry{webinterface}{
	name={Webinterface},
	description={Ein Web Interface ist ein System, durch welches Anwender mit dem Netz interagieren. Der Begriff Web Interface steht zumeist für grafische Oberflächen.}
}

\newglossaryentry{docker}{
	name={Docker},
	description={Docker ist eine Software, welche es ermöglicht Programme in einer abgeschnittenen Umgebung (genannt Container) laufen zu lassen. Das Erstellen dieser Umgebung und das Installieren und Laufen des Programms darin, gestaltet sich hierbei sehr einfach.\cite{dockerEngineOverview}}
}

\newglossaryentry{dcompose}{
	name={Docker Compose},
	description={Docker Compose ist eine Erweiterung von Docker. Mit ihr kann man multiple Container gleichzeitig aufbauen, womit es ermöglicht wird komplexe Infrastruktur - wie in Refundable benötigt - einfach aufzubauen, zu reproduzieren und letztlich auf die Computer, auf denen während der Produktion die Infrastruktur laufen wird, zu liefern.\cite{dockerComposeOverview}}
}

\newglossaryentry{relDb}{
	name={relationale Datenbank},
	description={Datenbank, wo ein Typ von Daten durch eine Tabelle repräsentiert wird. Die Anzahl der Spalten ist hierbei für jeden Datensatz konstant.}
}

\newglossaryentry{nosqlDb}{
	name={nicht-relationale Datenbank (\textmd{auch} NoSQL Datenbank)},
	description={Datenbank die von dem Konzept, dass Daten durch eine Tabelle repräsentiert wird, abweicht. Hierbei gibt es sehr viele verschiedene Ansätze, die von einer Datenbank-Software zur anderen verschieden sind.}
}

\newglossaryentry{json}{
	name={JSON},
	description={Bei JSON (JavaScript Object Notation) handelt es sich um ein kompaktes textbasiertes Datenformat, welches für den Datenaustausch zwischen Schnittstellen entwickelt wurde.\cite{rfc4627}}
}

\newglossaryentry{yaml}{
	name={YAML},
	description={YAML (YAML Ain't Markup Language) ist eine Markup-Sprache, die zur Beschreibung von Daten genutzt wird. Hierbei zeichnet sich YAML genau dabei aus, dass es nicht nur für die Maschine, sondern auch für den Menschen gut lesbar ist.}
}

\newglossaryentry{xml}{
	name={XML},
	description={XML (Extensible Markup Language) ist eine Markup-Sprache, die zur Beschreibung von Daten genutzt wird. Wobei die Datenstruktur von XML über ein Schema verifiziert werden kann.\cite{Nurseitov}}
}

\newglossaryentry{konsistenz}{
	name={Konsistenz},
	description={Konsistenz bedeutet die Einhaltung von Regeln. Im Zusammenhang mit Daten in einer Datenbank ist gemeint, dass die Daten die im \gls{dbms} gespeichert sind, eingehaltet werden müssen.}
}

\newglossaryentry{verfugbarkeit}{
	name={Verfügbarkeit},
	description={Verfügbarkeit im Zusammenhang mit einem Dienst bedeutet, dass jener Dienst zu einer bestimmten Zeit erreichbar ist. Eine hohe Verfügbarkeit bedeutet, dass der Dienst (fast) immer erreichbar ist.}
}

\newglossaryentry{https}{
	name={HTTPS},
	description={Unter HTTPS (Hypertext Transfer Protocol Secure) versteht sich ein Kommunikationsprotokoll, welches über Verschlüsselung durch Zertifikate eine Verbindung zwischen Server und Client im Web sicherer macht.}
}
