\newglossaryentry{designpattern}{
	name={Entwurfsmuster},
	description={Eine Vorlage, wie man ein Programm oder Software intern strukturiert.}
}

\newglossaryentry{jsframework}{
	name={JavaScript Framework},
	description={Ein \Gls{framework} für \Gls{js}}
}

\newglossaryentry{mvvm}{
	name={MVVM},
	description={MVVM (Model-View-ViewModel) ist ein Entwurfsmuster, welches eine Variante des \Gls{mvc}-Patterns ist}
}

\newglossaryentry{mvc}{
	name={MVC},
	description={MVC (Model-View-Controller) ist ein Entwurfsmuster, welches die Logik eines Programms von dem Interface(View) trennt}
}

\newglossaryentry{mv*}{
	name={MV*},
	description={MV* (Model-View-*) ist eine Zusammenfassung aller Model-View Patterns. \Gls{mvvm} und \Gls{mvc} fallen beide in diesem Schema hinein}
}


\newglossaryentry{vue}{
	name={VueJS},
	description={VueJs ist ein \Gls{jsframework} entwickelt von Evan You}
}

\newglossaryentry{angular}{
	name={Angular},
	description={Angular ist ein \Gls{jsframework} entwickelt von Google}
}

\newglossaryentry{react}{
	name={ReactJS},
	description={ReactJS ist ein \Gls{jsframework} entwickelt von Facebook}
}

\newglossaryentry{opensource}{
	name={Open-Source},
	description={Open-Source bedeutet, dass der Sourcecode öffentlich ist und von Dritten einsehbar ist}
}