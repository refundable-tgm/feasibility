\section{Projektleitung \& Frontend - responsives Webdesign}
	\subsection{Überblick}
	In diesem Teil werden die Anforderungen des \Gls{frontend}s geschildert. Da es mehrere Möglichkeiten gibt das \Gls{frontend} zu realisieren werden hier zwei wesentliche Methoden verglichen. Zum Einen die Variante \Gls{html}, \Gls{css} und \Gls{js} zu verwenden, zum Anderen eine \Gls{webapplikation} basierend auf \Gls{java} mit Hilfe eines \Gls{full stack framework}s. Ebenfalls werden hier verschiedene \Gls{css} \Gls{framework}s verglichen. Anschließend werden die zwei verschiedenen Methoden verglichen und die Beste ausgewählt. Zu guter Letzt wird das Design für die Zielgruppe analysiert, da Refundable explizit für Lehrer maßgeschneidert wird.
	\subsection{Projektmanagement}
	\subsection{CSS Frameworks}
	\Gls{css} \Gls{framework}s
		\subsubsection{Bootstrap}
		\subsubsection{Materialize}
		\subsubsection{ZURB Foundation}
		\subsubsection{Tailwind CSS}
		Tailwind \Gls{css}
		\subsubsection{SASS}
		\subsubsection{Vanilla CSS}
		\Gls{vanilla} \Gls{css}
		\subsubsection{Vergleich}
	\subsection{Full Stack Framework}
	\Gls{full stack framework}
	\subsection{Vergleich HTML und CSS mit Java}
	Vergleich \Gls{html} und \Gls{css} mit \Gls{java}
	\subsection{Zielgruppenorientiertes Design}
	\subsection{Fragestellungen}
