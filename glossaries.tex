% Akronyme

\newacronym{lul}{LuL}{Lehrer*innen}



% Glossar
\newglossaryentry{pdf}{
	name={PDF},
	description={\enquote{PDF ist ein universelles Dateiformat, das besonders für das elektronische Publizieren und in der Druckvorstufe eingesetzt wird}}
}

\newglossaryentry{webinterface}{
	name={Webinterface},
	description={\enquote{Ein Web Interface ist ein System, durch welches Anwender mit dem Netz interagieren. Der Begriff Web Interface steht zumeist für grafische Oberflächen.}}
}

\newglossaryentry{docker}{
	name={Docker},
	description={\enquote{Docker ist eine Software, welche es ermöglicht Programme in einer abgeschnittenen Umgebung (genannt Container) laufen zu lassen. Das Erstellen dieser Umgebung und das Installieren und Laufen des Programms darin, gestaltet sich hierbei sehr einfach.\cite{dockerEngineOverview}}}
}

\newglossaryentry{dcompose}{
	name={Docker Compose},
	description={\enquote{Docker Compose ist eine Erweiterung von Docker. Mit ihr kann man multiple Container gleichzeitig aufbauen, womit es ermöglicht wird komplexe Infrastruktur - wie in Refundable benötigt - einfach aufzubauen, zu reproduzieren und letztlich auf die Computer, auf denen während der Produktion die Infrastruktur laufen wird, zu liefern.\cite{dockerComposeOverview}}}
}

\newglossaryentry{frontend}{
	name={Frontend},
	description={Das Frontend ist die Oberfläche einer Website - also das was zu sehen ist.}
}

\newglossaryentry{backend}{
	name={Backend},
	description={Das Backend ist die Funktionalität im Hintergrund des \Gls{frontend}s.}
}

\newglossaryentry{css}{
	name={CSS},
	description={Cascading Style Sheet. Zum umgestalten und verschönern der Weboberfläche.}
}

\newglossaryentry{html}{
	name={HTML},
	description={Hypertext Markup Language. Die Grundbausteine bzw. Struktur der Weboberfläche. Zum Beispiel Text oder eine Eingabefläche.}
}

\newglossaryentry{js}{
	name={JavaScript},
	description={Java Script. Bietet einen Rahmen an Funktionalität und Animationen in Verbindung mit \Gls{html} und \Gls{css}.}
}

\newglossaryentry{framework}{
	name={Framework},
	description={Kann als Baukasten gesehen werden. Bietet Möglichkeiten um die normalen Vorhergehensweisen zu kürzen bzw. vereinfachen.}
}

\newglossaryentry{java}{
	name={Java},
	description={Eine Programmiersprache der Firma Oracle, wird meist für Desktopapplikationen bzw. für das \Gls{backend}.}
}

\newglossaryentry{jsframework}{
	name={JavaScript Framework},
	description={Ein \Gls{framework} für \Gls{js}.}
}

\newglossaryentry{full stack framework}{
	name={Full Stack Framework},
	description={Ein \Gls{framework} für \Gls{backend} als auch \Gls{frontend}.}
}

\newglossaryentry{vanilla}{
	name={Vanilla},
	description={Ein anderer Begriff für Basisausführung.}
}

\newglossaryentry{webapplikation}{
	name={Webapplikation},
	description={Wie eine Programm auf dem PC, nur dass das Programm nicht auf dem PC installiert wird, sondern im Internet aufgerufen und geladen wird.}
}

\newglossaryentry{design-pattern}{
	name={Design-Pattern},
	description={Eine Vorlage, wie man ein Programm oder Software intern strukturiert.}
}
